\chapter{Controle e monitoramento de verificação}

É importante notar que o projeto analisado por esse trabalho já foi congelado 
e não está em desenvolvimento, mas ainda assim é possível prever soluções e 
como controlar a execução das correções propostas.

Os autores indicaram intenção de voltar o projeto para evolução na Disciplina de 
Manutenção e Evolução de software. Dessa maneira foram estipuladas duas estratégias 
de controle:

\section{Sem Prazos}

Se a equipe preferir seguir com controle sem prazos específicos de correção,
mas deixar a correção como uma opção do backlog, A sugestão desse trabalho é
que se utilize a funcionalidade do \textbf{GitHub} de delegação de issues. Dessa forma
a equipe pode planejar essas correções em suas textit{sprints} de desenvolvimento.


\section{Com Prazos}

Caso a equipe preferir solicitar as correções com prazos específicos para o processo
a sugestão desse trabalho é que se utilize o \textbf{RedMine}, com sua funcionalidade
de delegação de tarefas e gerenciamento de atividades e datas.
