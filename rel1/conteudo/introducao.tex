\chapter{Introdução}

Medição e Análise é uma área muito importante do ambiente computacional, talvez a mais importante por lidar com a parte do cálculo de investimentos e custos dos softwares.

 “Medição de software é uma avaliação quantitativa de qualquer aspecto dos processos e produtos de software, que permite seu melhor entendimento e, com
isso, auxilia o planejamento, controle e melhoria do que se produz e de como é produzido"~\cite{rocha2012mediccao}

Apesar do conceito, o grupo visa medir a qualidade do processo de engenharia de requisitos com base na aderência aos modelos de maturidade.

Segundo o autor Albuquerque, A melhoria de processos de software pode estar relacionada a: (i) galgar
níveis mais altos de maturidade (melhoria vertical)e/ou (ii) realizar mudanças visando a uma maior adequação às necessidades da organização ou melhorias no
desempenho dos processos (melhoria horizontal).\cite{albuquerque2008avaliaccao}

Nos dois casos, a melhoria dos
processos deve estar associada a objetivos de melhoria que visam principalmente:
(i) entender as características dos processos existentes e os fatores que afetam o
seu desempenho; (ii) planejar e implementar ações que modifiquem o processo
para atender melhor às necessidades de negócio; e (iii) avaliar os impactos e os
benefícios obtidos com as mudanças realizadas nos processos.\cite{albuquerque2008avaliaccao}

Após estudar a técnica de Medição chamada GQM, e após propor o plano deste, a equipe referente a este relatório iniciou a confecção do que havia sido planejado.
