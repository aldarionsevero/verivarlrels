\chapter{Introdução}

A verificação de documentos é necessária para assegurar que esse atenda as necessidades e cumpra as especificações. É ideal realizar verificações ao longo de todo o processo, principalmente em documentos e definições que fazem parte da elaboração inicial de projetos de \textit{Software}. Dessa maneira, pode-se evitar perdas significativas ao identificar erros logo no início do projeto.

A Engenharia de Software possui várias atividades para garantir a qualidade tanto dos processos, artefatos e do \textit{software} em si. Na análise desse trabalho utilizou-se técnicas estáticas de verificação. Essas técnicas referem-se a Análise ou checagem do tanto do software em si, quanto de seus documentos ou processos de maneira mais formal e documentada, como por exemplo a técnica de Inspeção.

Realizar verificações de software e seus documentos podem assegurar benefícios para a organização como um todo, visto que o retrabalho configura cerca de 40\% do esforço gasto no projeto~\cite{}. 

A maneira mais eficaz de reduzir esses números é detectar erros logo no início do projeto. De acordo com~\cite{} a redução de esforço é maior quando detectado problemas em Modelos de maturidade, arquiteturas de software e gerência de riscos. 

