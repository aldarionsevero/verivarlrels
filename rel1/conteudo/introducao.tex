\chapter{Introdução}

A verificação de documentos é necessária para assegurar que este atenda as necessidades e cumpra as especificações. É ideal realizar verificações ao longo de todo o processo, principalmente em documentos e definições que fazem parte da elaboração inicial de projetos de Software. Dessa maneira, pode-se evitar perdas significativas ao identificar erros logo no início do projeto.
A Engenharia de Software possui várias atividades para garantir a qualidade tanto dos processos, artefatos e do software em si. Os dois tipos de verificações são: 

Estáticas: Técnicas de Verificação estáticas referem-se a Análise ou checagem do software, seus documentos ou processos de maneira mais formal e documentada, como a técnica da Inspeção.

Realizar verificações de software e seus documentos podem assegurar benefícios para a organização como um todo, visto que o retrabalho configura cerca de 40\% do esforço gasto no projeto~\cite{}. 
A maneira mais eficáz de reduzir esses números é detectar erros logo no início do projeto. De acordo com~\cite{} a redução de esforço é maior quando detectado problemas em Modelos de Maturidade, Arquiteturas de software e gerência de riscos. 

